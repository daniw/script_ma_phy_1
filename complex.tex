% coding:utf-8

%----------------------------------------
%FOSAMATH, a LaTeX-Code for a mathematical summary for basic analysis
%Copyright (C) 2013, Daniel Winz, Ervin Mazlagic, Adrian Imboden, Philipp Langer

%This program is free software; you can redistribute it and/or
%modify it under the terms of the GNU General Public License
%as published by the Free Software Foundation; either version 2
%of the License, or (at your option) any later version.

%This program is distributed in the hope that it will be useful,
%but WITHOUT ANY WARRANTY; without even the implied warranty of
%MERCHANTABILITY or FITNESS FOR A PARTICULAR PURPOSE.  See the
%GNU General Public License for more details.
%----------------------------------------

% coding:utf-8
\section{Komplexe Zahlen}
Die Gleichung $x^2 + 1 = 0$ hat in $\mathbb{R}$ keine Lösung. 
\[ x^2 + 1 = 0 \Leftrightarrow x^2 = -1 \qquad ; \text{aber} x^2 \geq 0 \forall x \in \mathbb{R} \]
Wir erweitern daher den Bereich der reellen Zahlen $\mathbb{R}$ durch Hinzugabe neuer Zahlen so zu einem Zahlenbereich $\mathbb{C}$, dass die Gleichung $x^2 = -1$ in $\mathbb{C}$ lösbar ist. $\mathbb{C}$ soll eine Zahl enthalten, deren Quadrat $-1$ ist. Diese Zahl heisst imaginäre Einheit und wird mit $j$ bezeichnet: 
\[ \boxed{j^2 = -1} \qquad \text{Imaginäre Einheit} \]
Es soll also gelten: $\mathbb{R} \cup \{j\} \in \mathbb{C}$. 
Andererseits sollen die in $\mathbb{R}$ geltenden Körperaxiome auch in $\mathbb{C}$ gelten. 
Daraus folgt: $(a + b j) \in \mathbb{C} \forall a, b \in \mathbb{R}$. 
\subsubsection{Definition:}
\begin{framed}\noindent
Die Summe $a + b j~(a, b, \in \mathbb{R})$ heissen komplexe Zahlen. $a$ heisst Realteil, $b$ Imaginärteil. Die Produkte der From $b j~(b \in \mathbb{R})$ heissen imaginäre Zahlen. Die Menge aller komplexen Zahlen wird mit $\mathbb{C}$ bezeichnet. 
\end{framed}
In Zeichen: 
\\$ \mathbb{C}:=\{a + b j \quad | \quad a, b \in \mathbb{R}\} $
\\Sei $z = a + b j$ eine komplexe Zahl. Man nennt
\\$a = $ Re $ z$ : Realteil von $z$. 
\\$b = $ Im $ z$ : Imaginärteil von $z$. 
\[ \boxed{\overline{z} := a - b j} \qquad%
\text{heisst die zu $z = a + b j$ konjugiert komplexe Zahl. } \]

\subsubsection{Beispiele:}
Komplexe Zahlen: $3 + 5 j, -7 + \frac{3}{4} j, 6 - \pi j$\\
Imaginäre Zahlen: $5 j, \frac{\pi}{2} j, -7 j$\\
Sei $z = 3 + 5 j$, dann ist: $3 =$ Re $z, 5 =$ Im $z, \overline{z} = 3 - 5 j$. \\
$\overline{\overline{z}} = 3 + 5 j = z$\\
Für $a \in \mathbb{R}$ gilt: $a = a + 0 j$, also $a \in \mathbb{R}$. Somit: 
\[ \boxed{\mathbb{R} \subset \mathbb{C}} \qquad \text{Die Menge der komplexen Zahlen enthält $\mathbb{R}$} \]
Für $b \in \mathbb{R}$ gilt: $b j = 0 + b j$, also $b j \in \mathbb{C}$. \\
Insbesondere ist $j \in \mathbb{C}$. 

\subsection{Bereich der komplexen Zahlen als Körper}
Gleichheit: 
\[ \boxed{a_1 + b_1 j = a_2 + b_2 j \Leftrightarrow a_1 = a_2\text{ und }b_1 = b_2} \]
Beweis: \\
$a_1 + b_1 j = a_2 + b_2 j \Rightarrow a_1 - a_2 = (b_2 - b_1) j$\\
Wäre $b_2 \neq b_1 \Rightarrow b_2 - b_1 \neq 0 \Rightarrow j = \frac{a_1 - a_2}{b_2 - b_1} \in \mathbb{R}$: unmöglich\\
Somit ist $b_2 = b_1 \Rightarrow b_2 - b_1 = 0 \Rightarrow a_1 - a_2 = 0 \Rightarrow a_1 = a_2$

\subsection{Addition und Subtraktion}
\[ \boxed{(a_1 + b_1 j) \pm (a_2 + b_2 j) = (a_1 \pm a_2) + (b_1 \pm b_2) j} \]
Beispiel: \\
$(1 + 2 j) + (3 - 5 j) = (1 + 3) + (2 - 5) j = 4 - 3 j $
