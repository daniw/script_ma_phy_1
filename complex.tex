% coding:utf-8

%----------------------------------------
%FOSAMATH, a LaTeX-Code for a mathematical summary for basic analysis
%Copyright (C) 2013, Daniel Winz, Ervin Mazlagic, Adrian Imboden, Philipp Langer

%This program is free software; you can redistribute it and/or
%modify it under the terms of the GNU General Public License
%as published by the Free Software Foundation; either version 2
%of the License, or (at your option) any later version.

%This program is distributed in the hope that it will be useful,
%but WITHOUT ANY WARRANTY; without even the implied warranty of
%MERCHANTABILITY or FITNESS FOR A PARTICULAR PURPOSE.  See the
%GNU General Public License for more details.
%----------------------------------------

% coding:utf-8
\section{Komplexe Zahlen}
Die Gleichung $x^2 + 1 = 0$ hat in $\mathbb{R}$ keine Lösung. 
\[ x^2 + 1 = 0 \Leftrightarrow x^2 = -1 \qquad ; \text{aber} x^2 \geq 0 \forall x \in \mathbb{R} \]
Wir erweitern daher den Bereich der reellen Zahlen $\mathbb{R}$ durch Hinzugabe neuer Zahlen so zu einem Zahlenbereich $\mathbb{C}$, dass die Gleichung $x^2 = -1$ in $\mathbb{C}$ lösbar ist. $\mathbb{C}$ soll eine Zahl enthalten, deren Quadrat $-1$ ist. Diese Zahl heisst imaginäre Einheit und wird mit $j$ bezeichnet: 
\[ \boxed{j^2 = -1} \qquad \text{Imaginäre Einheit} \]
Es soll also gelten: $\mathbb{R} \cup \{j\} \in \mathbb{C}$. 
Andererseits sollen die in $\mathbb{R}$ geltenden Körperaxiome auch in $\mathbb{C}$ gelten. 
Daraus folgt: $(a + b j) \in \mathbb{C} \forall a, b \in \mathbb{R}$. 
\subsubsection{Definition:}
\begin{framed}\noindent
Die Summe $a + b j~(a, b, \in \mathbb{R})$ heissen komplexe Zahlen. $a$ heisst Realteil, $b$ Imaginärteil. Die Produkte der From $b j~(b \in \mathbb{R})$ heissen imaginäre Zahlen. Die Menge aller komplexen Zahlen wird mit $\mathbb{C}$ bezeichnet. 
\end{framed}\noindent
In Zeichen: 
\\$ \mathbb{C}:=\{a + b j \quad | \quad a, b \in \mathbb{R}\} $
\\Sei $z = a + b j$ eine komplexe Zahl. Man nennt
\\$a = $ Re $ z$ : Realteil von $z$. 
\\$b = $ Im $ z$ : Imaginärteil von $z$. 
\[ \boxed{\overline{z} := a - b j} \qquad%
\text{heisst die zu $z = a + b j$ konjugiert komplexe Zahl. } \]

\subsubsection{Beispiele:}
Komplexe Zahlen: $3 + 5 j, -7 + \frac{3}{4} j, 6 - \pi j$\\
Imaginäre Zahlen: $5 j, \frac{\pi}{2} j, -7 j$\\
Sei $z = 3 + 5 j$, dann ist: $3 =$ Re $z, 5 =$ Im $z, \overline{z} = 3 - 5 j$. \\
$\overline{\overline{z}} = 3 + 5 j = z$\\
Für $a \in \mathbb{R}$ gilt: $a = a + 0 j$, also $a \in \mathbb{R}$. Somit: 
\[ \boxed{\mathbb{R} \subset \mathbb{C}} \qquad \text{Die Menge der komplexen Zahlen enthält $\mathbb{R}$} \]
Für $b \in \mathbb{R}$ gilt: $b j = 0 + b j$, also $b j \in \mathbb{C}$. \\
Insbesondere ist $j \in \mathbb{C}$. 

\subsection{Bereich der komplexen Zahlen als Körper}
Gleichheit: 
\[ \boxed{a_1 + b_1 j = a_2 + b_2 j \Leftrightarrow a_1 = a_2\text{ und }b_1 = b_2} \]
Beweis: \\
$a_1 + b_1 j = a_2 + b_2 j \Rightarrow a_1 - a_2 = (b_2 - b_1) j$\\
Wäre $b_2 \neq b_1 \Rightarrow b_2 - b_1 \neq 0 \Rightarrow j = \frac{a_1 - a_2}{b_2 - b_1} \in \mathbb{R}$: unmöglich\\
Somit ist $b_2 = b_1 \Rightarrow b_2 - b_1 = 0 \Rightarrow a_1 - a_2 = 0 \Rightarrow a_1 = a_2$

\subsection{Addition und Subtraktion}
\[ \boxed{(a_1 + b_1 j) \pm (a_2 + b_2 j) = (a_1 \pm a_2) + (b_1 \pm b_2) j} \]
Beispiel: \\
$(1 + 2 j) + (3 - 5 j) = (1 + 3) + (2 - 5) j = 4 - 3 j $

\subsection{Multiplikation}
$(a_1 + b_1 j) (a_2 + b_2 j) = a_1 a_2 + a_1 b_2 j + a_2 b_1 j + b_1 b_2 j^2$\\
Wegen $j^2 = -1$ folgt daraus: 
\[ \boxed{(a_1 + b_1 j) (a_2 + b_2 j) = (a_1 a_2 - b_1 b_2) + a_1 b_2 + a_2 b_1) j} \]
Beispiel: \\
$(2 + j) (5 - 3 j) = 10 - 6 j + 5 j - 3 j^2 = 13 - j$

\subsection{Potenzen}
Potenzen von $j$: \\
$j^1 = j, \quad j^2 = -1, \quad j^3 = -j, \quad j^4 = +1, \quad j^5 = j, \quad j^6 = -1, \quad \dots$\\
Allgemein für $n \in \mathbb{N}$: 
\[ \boxed{j^{4n} = 1, \quad j^{4n+1} = j, \quad j^{4n+2} = -1, \quad j^{4n+3} = -j} \]
Beispiele: \\
$j^{42} = -1, \quad j^{63} = -j, \quad j^{1000} = 1$

Potenzen komplexer Zahlen berechnet man mit Hilfe des binomischen Lehrsatzes. \\
Beispiel: \\
$(2 + 3 j)^5 = 2^5 + 5 \cdot 2^4 (3 j) + 10 \cdot 2^3 (3 j)^2 + 10 \cdot 2^2 (3 j)^3 + 5 \cdot 2 \cdot (3 j)^4 + (3 j)^5$\\
$ = 32 + 240 j + 720 j^2 + 1080 j^3 + 810 j^4 + 243 j^5$\\
$= 32 + 240 j - 720 - 1080 j + 810 + 243 j = 122 - 597 j$

\section{Division}
\[ \boxed{\frac{a_1 + b_1 j}{a_2 + b_2 j} = \frac{(a_1 + b_1 j)(a_2 - b_2 j)}{(a_2 + b_2 j)(a_2 - b_2 j)} = \frac{a_1 a_2 + b_1 b_2}{{a_2}^2 + {b_2}^2} + \frac{a_2 b_1 - a_1 b_2}{{a_2}^2 + {b_2}^2} j} \]
Beispiele: \\
\[ \frac{3 - 4 j}{1 + 2 j} = \frac{(3 - 4 j)(1 - 2 j)}{1 - (2 j)^2} = \frac{-5 -10}{5} = -1 -2 j \]
\[ (3 + 2 j)^{-3} = \frac{1}{(3 + 2 j)^3}; \quad (3 + 2 j)^3 = 27 + 54 j - 36 - 8 j = -9 + 46 j \]
\[ \frac{1}{(3 + 2 j)^3} = \frac{1}{-9 + 46 j} = \frac{-9 + 46 j}{81 + 2116} = -\frac{9}{2197} - \frac{46}{2197} j \]

\subsubsection{Fazit}
Wir haben festgestellt, dass die vier Grundoperationen nicht aus der Menge der komplexen Zahlen herausführen. Man kann leicht nachrechnen, dass die Axiome des reellen Zahlenkörpers auch von den komplexen Zahlen erfüllt werden. 
\begin{framed}\noindent
  Die komplexen Zahlen erfüllen sämtliche Körperaxiome. 
\end{framed}\noindent
Daraus folgt: \\
Mit den komplexen Zahlen darf man formal so rechen wie mit den reellen Zahlen! 

\section{Fundamentalsatz der Algebra}
Der Körper der komplexen Zahlen enthält eine Zahl $j$, deren Quadrat $-1$ ist: $j^2 = -1$. Somit ist die Gleichung $x^2 + 1 = 0$ in $\mathbb{C}$ lösbar: $x_1 = j, \quad x_2 = -j$\\\\
Allgemein gilt der
\subsubsection{Fundamentalsatz der Algebra}
\begin{framed}\noindent
  Jede algebraische Gleichung n-uen Grades $a_n z^n + a_{n-1} z^{n-1} + \dots + a_1 z + a_0 = 0$ mit komplexen Koeffizienten $a_0, a_1, \dots, a_n$ hat mindestens eine Komplexe Lösung. 
\end{framed}\noindent
Beispiel: \\
$x^2 + px + q = 0 \qquad \text{ mit }p, q \in \mathbb{R} \subset \mathbb{C}$\\
$x_{1, 2} = \left\{ \begin{matrix}
  \dfrac{-p \pm \sqrt{p^2 - 4 q}}{2} \qquad & \text{wenn } p^2 - 4 q \geq 0\\\\
  \dfrac{-p \pm j \sqrt{4 q - p^2}}{2} \qquad & \text{wenn } p^2 - 4 q < 0
\end{matrix} \right.$

\section{Konjugiert - komplexe Zahlen}
Seien $z, z_1, z_2 \in \mathbb{C}$ mit $z = a + b j, \overline{z} = a - b j$, dann gelten folgende Rechenregeln: 
\[ \boxed{\begin{array}{l}
  z \cdot \overline{z} = a^2 + b^2 \geq 0 \qquad , \text{ also } z \cdot \overline{z} \in \mathbb{R}\\\\
  z \cdot \overline{z} = 0 \Leftrightarrow z = 0 + 0 j = 0\\\\
  z = \overline{z} \Leftrightarrow z \in \mathbb{R} (\text{ also } b = 0)\\\\
  z = -\overline{z} \Leftrightarrow z \text{ ist imaginär } (a = 0)\\\\
  \overline{z_1 \pm z_2} = \overline{z_1} \pm \overline{z_2} \qquad \overline{\left( \frac{z_1}{z_2} \right)} = \frac{\overline{z_1}}{\overline{z_2}}\\\\
  \overline{z_1 \cdot z_2} = \overline{z_1} \cdot \overline{z_2} \qquad \overline{\overline{z}} = z
\end{array}} \]
Die Division kan jetzt kürzer formuliert werden: \\
$\dfrac{z_1}{z_2} = \dfrac{z_1 \cdot \overline{z_2}}{z_2 \cdot \overline{z_2}} = \dfrac{z_1 \cdot \overline{z_2}}{{a_2}^2 + {b_2}^2}$\\\\
$\dfrac{1}{z} = \dfrac{\overline{z}}{z \cdot \overline{z}} = \dfrac{a - b j}{a^2 + b^2} = \dfrac{a}{a^2 + b^2} - \dfrac{b}{a^2 + b^2} j$\\\\
Falls $z\neq 0$ ( $a \neq 0$ oder $b \neq 0$ ) gilt also: 
\[ \boxed{z^{-1} = \frac{a - b j}{a^2 + b^2}} \]
\begin{framed}\noindent
  Ist $z = a + b j$ eine Lösung der Gleichung $a_n z^n + a_{n-1} z^{n - 1} + \dots + a_1 z + a_0 = 0$ mit reelleen Koeffizienten $a_0, a_1, \dots, a_n$, so ost auch $\overline{z} = a - b j$ eine Lösung. 
\end{framed}\noindent
Eine algebraische Gleichung n-ten Grades mit reellen Koeffizienten hat also entweder keine oder paarweise konjugiert-komplexe Lösungen. \\\\
Beweis: \\
Sei $z \in L \Leftrightarrow a_n z^n + \dots + a_1 z + a_0 = 0$\\
$a_n \overline{z}^n + a_{n-1} \overline{z}^{n-1} + \dots + a_1 \overline{z} + a_0 = $\\
$\overline{a_n} \cdot \overline{z^n} + \overline{a_{n-1}} \cdot \overline{z^{n-1}} + \dots + \overline{a_1} \cdot \overline{z} + \overline a_0 = $\\
$\overline{a_n \cdot z^n + a_{n-1} \cdot z^{n-1} + \dots + a_1 \cdot z + a_0} = \overline{0} = 0$\\\\
Beispiel: \\
$z^3 - 1 = 0$ \\
Prüfe nach, dass neben $z_1 = 1$ auch $z_2 = -\frac{1}{2} + \frac{sqrt{3}}{2} j$ und $z_3 = \overline{z_2} = -\frac{1}{2} - \frac{\sqrt{3}}{2} j$ Lösungen sind. \\\\
Bemerkungen: \\
Sind die Koeffizienten der Gleichung $a_n z^n + a_{n-1} z^{n-1} + \dots + a_1 z + a_0 = 0$ nicht alle reell, so müssen komplexe Lösungen nicht mehr konjugiert-komplex auftreten. \\\\
Beispiel: \\
$z^3 - 4 z^2 + (6 + j) z + (-3 - j) = 0$\\
Prüfe nach, dass die Zahlen $z_1 = 1, \quad z_2 = 1 + j, \quad z_3 = 2 - j$ Lösungen dieser Gleichung sind. 

\section{Die Gausssche Zahlenebene}
Die komplexen Zahlen können nicht wie die reellen Zahlen  linear angeordnet werden. \\
Beispiel: $j \neq 0$\\
Wäre $j > 0 \Rightarrow j \cdot j > 0 \cdot j \Rightarrow -1 > 0$: Widerspruch!\\
Wäre $j < 0 \Rightarrow -j > 0 \Rightarrow j \cdot (-j) < 0 \cdot (-j) \Rightarrow 1 < 0$: Widerspruch!
\begin{framed}\noindent
  Für komplexe Zahlen haben die Zeichen $<$ und $>$ keinen Sinn! Es gibt (ausser den reellen Zahlen) keine positiven oder negativen Zahlen. 
\end{framed}\noindent
Jeder komplexen Zahl $a + b j$ entspricht genau ein reelles Zahlenpaar $(a, b): z = a + b j \leftrightarrow (a, b) \in \mathbb{R}^2$. Die komplexen Zahlen können daher durch die Punkte (Ortsvektoren) der Ebene dargestellt werden: 
