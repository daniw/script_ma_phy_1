% coding:utf-8

%----------------------------------------
%FOSAMATH, a LaTeX-Code for a mathematical summary for basic analysis
%Copyright (C) 2013, Daniel Winz, Ervin Mazlagic, Adrian Imboden, Philipp Langer

%This program is free software; you can redistribute it and/or
%modify it under the terms of the GNU General Public License
%as published by the Free Software Foundation; either version 2
%of the License, or (at your option) any later version.

%This program is distributed in the hope that it will be useful,
%but WITHOUT ANY WARRANTY; without even the implied warranty of
%MERCHANTABILITY or FITNESS FOR A PARTICULAR PURPOSE.  See the
%GNU General Public License for more details.
%----------------------------------------

\chapter*{Über diese Arbeit}
Dies ist das Ergebnis einer Zusammenarbeit auf Basis freier Texte erstellt von Studierenden der Fachhochschule Luzern und ist unter der GPLv2 lizenziert. Der \TeX - bzw. \LaTeX -Code ist auf \url{github.com/daniw/fosamath} hinterlegt. Eine aktuelle PDF-Ausgabe steht auf \url{fosa.adinox.ch} zum Download bereit.

In dieser Formelsammlung sind die Inhalte des MATH-Moduls der HSLU-T\&A zusammengefasst.
%
\iftiboth
	Zudem sind in dieser Formelsamlung Tipps und Hinweise für die Bedienung des TI-89 und des TI-Nspire CAS enthalten. 
	\else
	\ifti
		Zudem sind in dieser Formelsammlung Tipps und Hinweise für die Bedienung des TI-89 enthalten. 
	\fi
	\ifnspire
		Zudem sind in dieser Formelsammlung Tipps und Hinweise für die Bedienung des TI-Nspire CAS enthalten. 
	\fi
\fi



%\ifti
%Zudem sind in dieser Formelsammlung Tipps und Hinweise für die Bedienung des TI-89 enthalten. 
%\fi
%\ifnspire
%Zudem sind in dieser Formelsammlung Tipps und Hinweise für die Bedienung des TI-Nspire CAS enthalten. 
%\fi
%Leider kann für nicht bestandene Prüfungen keine Haftung übernommen werden.

\section*{Danksagung}
An dieser Stelle möchten wir allen danken, die uns bei diesem Projekt unterstützt haben.
Einerseits sind dies alle Contributors auf dem Github-Repository fosamath und jene Studenten die
periodische Rückmeldungen gegeben haben.
Ein spezieller Dank geht dabei an unseren Dozenten Mario Amrein, welcher uns eine unschätzbare Hilfe war.
Er hat uns nicht nur bei der Erarbeitung und Überprüfung der Formelsammlung geholfen, sondern uns auch 
mit seinen Vorlesungen für die Mathematik begeistert und motiviert.


